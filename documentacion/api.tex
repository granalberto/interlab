\documentclass[titlepage,12pt]{article} 
\usepackage[utf8]{inputenc}
\usepackage[spanish]{babel}
\usepackage[pdftex]{graphicx}
%\usepackage[usenames]{color}
\usepackage{url}
%\pagestyle{headings}

\title{DIEA INTERLAB API\\
Manual del Programador}  
\author{Alberto Mijares\\  
\texttt{amijares@mcs.com.ve}\\
Mijares Consultoría y Sistemas SL\\
\texttt{http://www.mcs.com.ve}\\
\includegraphics[scale=.6]{logo-mijares.png}}
\date{\today}


\begin{document}

\maketitle

\tableofcontents

\newpage

\parskip=5mm

\section{Descripción General}\label{general}

Este documento describe el API \emph{(Application Programming
  Interface)} para la comunicación entre diferentes aplicaciones para
la gestión de laboratorios clínicos y el DIEA INTERLAB. Está orientado
a Ingenieros, Técnicos y profesionales informáticos con habilidades en
programación que deseen integrar sus aplicaciones o automatizar
ciertas operaciones del negocio.

Para los efectos de este documento, identificaremos los tres elementos
principales que actúan como protagonistas:
\begin{description}
  \item[Estación de Trabajo:] es cualquier computadora,
    conectada a la red de datos del laboratorio, desde donde es
    posible enviar solucitudes de análisis a los Analizadores; o bien,
    recibir los resultados de dichos análisis.
  \item[Interfaz:] es el dispositivo de software que sirve de
    puente de comunicación entre las Estaciones de Trabajo y los
    Analizadores. La interfaz a la que nos referimos en este documento
    es el DIEA INTERLAB (Dispositivo Interfaz para Equipos Analizadores).
  \item[Analizador:] es cualquiera de los equipos que analizan
    las muestras de laboratorio y generan un resultado. La lista de
    Analizadores soportados por la Interfaz está autocontenida en este
    manual.
\end{description}

La Interfaz ejecuta un webservice, encargado de manejar la
comunicación con las diferentes estaciones de trabajo a través del
protocolo HTTP\footnote{Hypertext Transfer Protocol}. Los detalles
para llevar a cabo esta comunicación se encuentran en la
sección~\ref{http}, página~\pageref{http}.

La comunicación entre la interfaz y cada analizador se llevará a cabo
de distintas formas, dependiendo de las facilidades que ofrezca cada
analizador. En algunos casos puede llevarse a cabo via serial, y en
otros casos puede ser a través de algún protocolo de la familia
IP\footnote{Internet Protocol}. De cualquier forma, la implementación
del webservice en la interfaz hace que esto sea transparente para el
desarrollador.

\section{Comunicación HTTP:\newline INTERFAZ $\longleftrightarrow$ Estación de Trabajo}\label{http}

\subsection{General}\label{http.general}

La interfaz permite la comunicación \underline{bidireccional} a través
de un webservice HTTP, a través
del cual se comunicará cualquier estación de trabajo. Para la llevar a
cabo dicha comunicación se debe implementar el protocolo HTTP o, en su
defecto, hacer uso de una librería que lo implemente en el lenguaje de
programación de su preferencia. Esta librería debe ser compatible con
el RFC 2616\footnote{Disponible en \url{http://www.rfc-editor.org/rfc/rfc2616.txt}}.

Existen tres operaciones básicas que podemos realizar cuando
interactuamos con la interfaz:
\begin{itemize}
\item Enviar un paciente a un analizador.
\item Solicitar los resultados disponibles de un analizador.
\item Cambiar la configuración de la red de la interfaz.
\end{itemize}

En esta sección se especifica cómo enviar un paciente y obtener los
resultados disponibles de un analizador. Cómo cambiar la configuración
de la interfaz se especifica en la sección \ref{netconfig}, página
\pageref{netconfig}.



Todas la operaciones se relizan enviando una petición HTTP al
webservice que se ejecuta en la interfaz, usando el método \textbf{GET}, puerto
\textbf{1080}, en el directorio~\textbf{cgi-bin}, al programa
\textbf{gw.cgi}; es decir, suponiendo que la IP de la interfaz es la
192.169.1.1, la URL base para cualquier operación sería

\texttt{http://192.168.1.1:1080/cgi-bin/gw.cgi?}

\noindent y luego se concatenan los parámetros requeridos para cumplir
con los requisitos de la operación.

Los parámetros del URL varían dependiendo del analizador con que se
comunique la estación de trabajo; con la excepción del parámetro
\textsf{lab}, que es requerido indistintamente, ya que su valor indica
al webservice con qué analizador se desea comunicar y qué librería
debe emplear para ejecutar la acción solicitada.

La lista de parámetros específicos para cada analizador, así como sus
posibles valores, se detallan a continuación:

\subsubsection{Analizador BT3000Plus}

Parámetros del URL y su descripción:

\begin{description}
  \item[\textsf{lab}] Determina la librería que se debe utilizar para
    enviar la petición al analizador. Para comunicarse con un
    analizador BT3000Plus, el valor de este parámetro debe ser la cadena
    \texttt{bt3000}.
  \item[\textsf{d}] El valor de esta variable corresponde al nombre
    del puerto serial que utiliza la interfaz para comunicarse con el
    analizador. En la mayoría de los casos, será la cadena de texto
    \texttt{ttyS0}\footnote{Nombre del dispositivo asociado con el
      primer puerto serial (DB9 - RS232) en sistemas GNU/Linux}, pero
    puede variar de acuerdo a ciertas características de la
    instalación. Se determina en el proceso de instalación de la
    interfaz.


  \item[\textsf{s}] Su valor debe ser una cadena de texto y representa
    una instrucción de acuerdo al tipo de operación que se desee
    realizar. Solo es posible realizar dos tipos de operaciones:
    enviar un paciente o consultar los resultados disponibles. En el
    caso del envío de pacientes, la cadena contiene todos los datos
    necesarios como código del paciente, códigos de los análisis a
    realizar, etc. El procedimiento para su construcción se detalla en
    más adelante esta sección. Cuando se consultan los resultados
    disponibles, el valor de este parámetro consiste en un solo
    caracter, como veremos más adelante.
\end{description}

Se debe tener en cuenta que si el valor de alguno de los parámetros
que se transmiten es más corto que lo requerido por el protocolo, debe
agragarse un espacio \textbf{antes} con el fin de cumplir con la
especificación. Por ejemplo, el código de cada tipo de análisis debe
tener una longitud de 4 caracteres; por ende, para enviar el código
\textsf{GLY}, se debe agregar un espacio para alzanzar la longitud
requerida como en \fbox{\textsf{ GLY}}.

La respuesta de la interfaz depende de dos factores: a) el tipo de
operación que se realizó y b) el resultado de la operación. Es muy
importante que la evaluación de los resultados se haga, en una primera
instancia, basada en el valor de las cabaceras \textsf{HTTP},
específicamente el \textsf{STATUS~CODE}; de esta forma estaremos
haciendo un uso adecuado del protocolo y liberamos de carga
innecesaria la aplicación que se ejecuta en la estación de
trabajo. También debemos señalar que el webservice está programado
para enviar una respuesta \emph{sin cuerpo} en un escenario
particular, por lo que la única forma de determinar qué tipo de
respuesta estamos recibiendo es evaluando el \textsf{STATUS~CODE}
recibido.

\paragraph{Transmisión del Paciente} 

Para construir la cadena de texto que se debe enviar al analizador, y
que será el valor del parámetro \textsf{s} en la URL, se deben seguir
las reglas que se explican a continuación y se ilustran mejor en la
siguiente tabla.

  \noindent\makebox[\width][r]{Código del paciente}\dotfill \mbox{15
    caracteres} \\
  \makebox[\width][r]{Tipo de lista}\dotfill
  \mbox{\textsf{T} para rutina o \textsf{R} para STAT}
  \\
  \makebox[\width][r]{Tipo de suero}\dotfill \mbox{\textsf{S} para
    suero y \textsf{U} para orina} \\
  \makebox[\width][r]{Indicar si
    el paciente es un clon}\dotfill \mbox{\textsf{Y} para si y
    \textsf{N} para no} \\
  \makebox[\width][r]{Posición en el
    contenedor}\dotfill \mbox{00 si es desconocido}
  \\
  \makebox[\width][r]{Número de test a ser ejecutados}\dotfill
  \mbox{del 01 al 99} \\
  \makebox[\width][r]{Códigos de tests a ser
    ejecutados}\dotfill \mbox{4 caracteres} \\


Si el código del paciente tiene menos de 15 caracteres, deben ser
completados con ceros a la izquierda. El tipo de lista es un
caracter. El tipo de suero es un caracter. El indicador de clon es un
caracter. La posición del contenedor se expresa con dos caracteres, y
debe ser 00 si no se desea especificar. La cantidad de pruebas a ser
ejecutado son dos caracteres numéricos. Los códigos de test a ser
ejecutados deben ser 4 caracteres, y completar con espacios en blanco
a la izquierda en los casos que lo requiera.

\fbox{\emph{Todo caracter no numérico debe estar en mayúscula.}}

Si la comunicación es exitosa, entonces la interfaz responde con la
cadena de texto ``Aceptado en la posicion'', seguido de un número que
identifica la posición donde se inserta el paciente. Por ejemplo:
``Aceptado~en~la~posicion~1''. El \textsf{STATUS~CODE} en este caso es
``\textsf{200~OK}''.

En caso de que la comunicación no sea exitosa La interfaz responde con
una cadena de texto explicando el error. Adicionalmente, el
\textsf{STATUS~CODE} corresponde a uno de los siguientes valores:

  \noindent\makebox[\width][r]{\textsf{412 Precondition Failed}}\dotfill \mbox{Indica que ocurrió un error}\\
  \makebox[\width][r]{\textsf{504 Gateway Timeout}}\dotfill \mbox{El analizador no responde}

Para enviar un paciente con código 00000000000001, tipo
de suero y análisis GLY, BUN y FEAL a las lista STAT, debe enviar la
siguiente secuencia de caracteres: ``\textsf{000000000000001RSN0003 GLY BUNFEAL}''. 

Siguiendo el mismo ejemplo, el URL que se debe contruir para
comunicarse con el webservice, suponiendo que la dirección IP de la
interfaz es 192.168.1.1, sería el siguiente:

\noindent{{\url{...gw.cgi?lab=bt3000&d=ttyS0&s=000000000000001RSN0003%20GLY%20BUNFEAL}}}

\emph{Nota Bene: en este ejemplo, los espacios en blanco han sido
  representados de acuerdo a la codificación URL (\%20). Debe asegurarse de
  que la librería utilizada para la comunicación http maneje la
  codificación URL adecuadamente, o hacerlo de forma manual en caso
  contrario.}



\paragraph{Recepción de Resultados}\par

Hay tres instrucciones para la recepción de resultados; es decir, tres
posibles valores para el parámetro \textsf{s}:

\begin{center}
  \begin{tabular}{|c|p{10cm}|}
    \hline
    \textsf{R} & Recepción del siguiente informe disponible \\
    \hline
    \textsf{L} &
  Recepción del último informe enviado (en caso de problemas con la
  recepción) \\
  \hline 
  \textsf{A} & Recepción del primer informe disponible (en
  caso de que se desee recibir nuevamente todos los informes) \\
  \hline
\end{tabular}
\end{center}

Como respuesta a una de estas tres instrucciones, el webservice envía
el informe requerido (si está disponible) y un \textsf{STATUS~CODE}
``\textsf{200 OK}''; en caso contrario, podemos obtener los siguientes
valores:

\begin{description}
  \item[\textsf{204 No Content}] Indica que no hay resultados
    disponibles. Este es el único escenario en el que no recibimos
    contenido en el cuerpo de la respuesta. De esta forma, solo
    debemos validar las cabeceras HTTP.
  \item[\textsf{412 Precondition Failed}] Se envió una consulta inválida.
  \item[\textsf{504 Gateway Timeout}] El analizador no responde.
  \item[\textsf{500 Internal Server Error}] El webservice retorna este
    código de error cuando no ha podido manejar la petición de manera
    adecuada, generalmente por no cumplirse con los requisitos para
    formular la petición. Se trata de una excepción no prevista en la
    programación.
\end{description}

Ante una consulta exitosa que haya recuperado resultados del
analizador, el webservice devuelve un objeto
\textsf{JSON}\footnote{Completamente documentado en
  \url{http://www.json.org/json-es.htlm}} como cuerpo del mensaje, que
contiene toda la información relativa a los resultados de las pruebas
disponibles en el analizador. La estructura del objeto es la siguiente:

\begin{minipage}{5cm}
\ttfamily 
\textbraceleft cod:``cadena'',\\
tl:``cadena'',\\
tm:``cadena''\\
ni:numero,\\
res:arreglo\}
\end{minipage}

El siguiente cuadro explica con el detalle el significado de cada par
en la estructura:
\begin{center}
\begin{tabular}{|c|c|c|} \hline
\textbf{nombre} & \textbf{tipo de dato} & \textbf{descripción}\\ \hline
cod & cadena & código del paciente \\ \hline
tl  & cadena & tipo de lista \\ \hline
tm  & cadena & tipo de muestra \\ \hline
ni  & numérico & número de informes \\ \hline
res & arreglo & resultados \\ \hline
\end{tabular}
\end{center}

Los resultados son presentados como un arreglo asociativo donde cada
par viene separado por comas y expresado de la forma

\fbox{\textsf{NAME:VALUE}} (nombre y valor separado por dos puntos)

\noindent donde \textsf{NAME} es el código del análisis y
\textsf{VALUE} representa el valor del resultado. Ambos valores son
una cadena de texto.
 
Para solicitar el siguiente informe disponible, debemos enviar
la siguiente petición al webservice:

\noindent{\url{http://192.168.1.1:1080/cgi-bin/gw.cgi?lab=bt3000&d=ttyS0&s=R}}

\section{Configuración de Red de la Interfaz}\label{netconfig}

Uno de los modos de operación del webservice funciona como un
mecanismo de configuración de la interfaz. Específicamente para
modificar la configuración de la red, de manera que pueda adecuarse al
entorno en el que se encuentra operando y forme parte de la misma red
lógica.

Los parámetros que se deben enviar en el URL para realizar esta
operación son los siguientes:

\begin{description}
\item[\textsf{lab}] Como ya hemos visto en la sección \ref{http},
  indica qué librería debe emplear el webservice. En este caso el
  valor debe ser la cadena de texto \textsf{ipconfig}.
\item[\textsf{ip}] Su valor corresponder con la nueva dirección IP a
  configurar. Por ejemplo: \textsf{172.16.12.200}.
\item[\textsf{mask}] Su valor corresponde con la nueva máscara de la
  red. Por ejemplo: \textsf{255.255.255.0}.
\item[\textsf{router}] Corresponde con la puerta de enlace por defecto de la red.
\end{description}

En caso de querer restaurar los valores por defectos de la interfaz,
basta con enviar el parámetro \textsf{lab} con su respectivo valor
\textsf{ipconfig} y omitir el resto de los parámetros. La interfaz
debe ser reiniciada para completar la operación.

\section{Código de Barras}\label{barcode}

Para la impresión de etiquetas se debe utilizar la
especificación para códigos de barra \textbf{CODE 39}. El dato a imprimir
debe ser el código del paciente (15 caracteres).

Esto debe ser vefiricado, dado que la documentación no es exacta al respecto.

\end{document}
